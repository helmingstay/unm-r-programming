\documentclass[letterpaper,12pt]{article}
\begin{document}

\section{Introduction}
Sanitary sewer blockages (SSB) cause widespread negative impacts, including
aesthetic degradation from odors, and property damage and environmental
degradation from sanitary sewage overflow (SSO).  In the U.S., where SSOs are
tracked by the U.S. Environmental Protection Agency \cite{epa2004local},
approximately half of SSOs were caused by blockages, with up to 75\% of SSOs
caused by blockages in the arid Southwest \cite{epa2004report}.
Consequently, prompt remediation of SSB is a high priority for municipalities,
and contributes to municipal sewer maintainence costs
\cite{maintainence-cost}.

Previous work has attributed SSBs primarily to roots, debris, and fats, oils,
and grease (FOG) \cite{epa2004report}.  In the U.S., 60-75\% of blockages
have fat, oil and grease (FOG) deposits as a contributory factor
\cite{Keener2008}, while vegetation intrusion is the chief cause of
blockages in Australia \cite{Marlow2011}.

As recognized contributors to SSB, FOG deposits have received considerable
attention. FOG deposits form in a saponification reaction between calcium soaps
and free fatty acids \cite{He2011}, chiefly from restaurants and
industrial sources \cite{Keener2008}.  Free fatty acids are insoluble in
water, and are transported in greasey effluent.  Many municipalities have
implemented policies to minimize FOG inputs into sanitary sewers
\cite{hassey2001grease, heckler2003best, parnell2005innovative,
bennett2006atlanta, tupper2008fog}.  Residential outreach is often increased
during the holiday season in an effort to minimize FOG inputs due to food
preparation \cite{tupper2008fog}.

Climate can influence blockage rate by affecting both vegetation and water
flow. \cite{Marlow2011}, for example, showed a correlation between sewer
blockage frequency and the Southern Oscillation Index (SOI) in eastern
Australia. The SOI reflects rainfall patterns in the region, with droughts
raising blockage risk by decreasing sewer flow volume and increasing
sedimentation. Low rainfall also promotes tree root development, which damage
pipes by intruding through joins and other weak points \cite{Desilva2011}.

Temperature is one potential driver of SSB that has received little study to
date. The viscosity of both water and FOGs decreases with decreasing
temperature. For a given pipe network, increased viscosity results in increased
frictional head loss \cite{romeo2002improved}. In addition, FOG effluent
can solidify at lower temperatures, causing overt blockages.

In this study we examine five years of SSB records from the City of Albuquerque
municipal sewer system.  We explore the relationship between air temperature,
sewage temperature, and the frequency of SSB.  We find that air temperature is
a useful proxy of sewage temperature, and that both air and sewage temperature
predict SSB frequency. Specifically, temperature predicts SSB events for which grease was a contributory factor, suggesting that cold weather increases the impact of FOG deposits. SSBs with other causes do not respond to temperature. These relationships shed light on mechanisms
of sewer blockage, and can potentially help municipalities anticipate time
periods of elevated sewer blockages using readily available atmospheric data.

\bibliographystyle{plainnat}
\bibliography{paper}
\end{document}
