\documentclass[english, letterpaper, 12pt]{article}
\usepackage{graphicx}
%% margins, can specify top, left, etc.
\usepackage[margin=1in]{geometry}
\usepackage[latin1]{inputenc}
%\usepackage[numbers]{natbib}
\usepackage{url}
\linespread{1.2}

\begin{document}

\begin{center}
\Large{R Crash Course}
~\\
\large{Summer 2014}
\large{Christian Gunning}
~\\
%\textbf{Name: \underline{\hspace{6cm}} }
\end{center}
 
\section*{What is it, and who's it for?}
The R Crash Course is an intensive workshop that focuses on data management and
visualization in R.   It is designed primarily for scientists, and is appropriate
for beginning and intermediate R users.   Using hands-on activities and your own
data, you will learn basic programming, gain experience in common data manipulation
tasks, and create your own publication-quality figures.

\section*{Background}
R is a statistical 
 analysis framework and programming language that is heavily used by
researchers in a wide range of fields.
Effective use of software tools like R can greatly enhance
research productivity and enable communication of results
through high-quality data visualization. New users often find R
challenging, intimidating, or downright confusing. 
The R Crash Course aims to accelerate the steep learning curve 
that many new R users encounter.  

\section*{Requirements}
The most important requirement of this workshop is time and a willingness to
work.  Partipants are expected to attend all sessions and spend 1-2 hours/day
individually on assignments.

Prior experience with R or a spreadsheet program like Excel is helpful but not
required.  Participants should be comfortable with basic
computer use, and must have their own computer.  Prior to the workshop, participants
must install R and Rstudio and select a dataset for workshop use.  Students are
encouraged to work in small groups on a shared dataset.

\clearpage
\section*{Schedule}
\subsection*{D1S1}
\begin{itemize}
    \item  Initial Evaluations
    \item  Syllabus
    \item  Introductions, datasets, research goals
    \item  Introduction to Rstudio - working with scripts
    \item  Reading in data
    \item  Project - read in your own data
\end{itemize}
\subsection*{D1S2}
\begin{itemize}
    \item  Quiz, Questions, Review
    \item  What is a data.frame?
    \item  Indexing
    \item  Subsetting
    \item  Objects, class, and mode.
    \item  Introduction to functions - summarizing data
    \item  Help pages
\end{itemize}

\subsection*{D2S1}
\begin{itemize}
    \item Quiz, Questions, Review
    \item Review assignment
    \item Functions
    \item Flow control
    \item Project - write your own function that uses a for loop to compute the
marginal mean/sd/nsamples for each group.
\end{itemize}
\subsection*{D2S2}
\begin{itemize}
    \item Quiz, Questions, Review
    \item Functions in functions - ddply
    \item Scoping and namespaces
    \item Debugging
    \item How to find help
\end{itemize}

\subsection*{D3S1}
\begin{itemize}
    \item Quiz, Questions, Review
    \item Plotting with ggplot - aesthetics and layers
    \item Project - plot data
\end{itemize}
\begin{itemize}
    \item Quiz, Questions, Review
    \item Advanced plotting -- facets, fine-tuning graphics
    \item Presentations - 10 min, 3 figures / group
    \item Final Review: where to go from here?
\end{itemize}

\section*{Assignments}
\subsection*{A1}
\begin{itemize}
    \item Using help pages
    \item Read in data
    \item Subset, summarize
    \item Install plyr and reshape2
\end{itemize}

\subsection*{A2}
\begin{itemize}
    \item Use Google to find a solution
    \item Debug a script - ddply function
    \item Write your own function for use in ddply, run analysis, explain
results
\end{itemize}

\end{document}
