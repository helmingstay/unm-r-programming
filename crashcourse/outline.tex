\documentclass[english, letterpaper, 12pt]{article}
\usepackage{graphicx}
%% margins, can specify top, left, etc.
\usepackage[margin=1in]{geometry}
\usepackage[latin1]{inputenc}
%\usepackage[numbers]{natbib}
\usepackage{url}
\linespread{1.2}

\begin{document}

\begin{center}
\Large{R Boot Camp}
~\\
\large{Fall 2014}
~\\
\large{Christian Gunning}

www.x14n.org
~\\
%\textbf{Name: \underline{\hspace{6cm}} }
\end{center}
 
\section*{What is it, and who's it for?}
The R Boot Camp is an intensive workshop that focuses on data management and
visualization in R.   It is designed primarily for graduate students who
frequently work with data, and is appropriate
for beginning and intermediate R users.   Participants will 
learn basic programming, gain experience in common data manipulation
tasks, and create publication-quality figures using hands-on activities and 
real-world datasets.

\section*{Background}
R has been called the {\em lingua franca} of statistics. R is a
programming language 
and statistical analysis framework that is used by
researchers in a wide range of scientific fields.

New users often find R
challenging and downright intimidating. 
The R Boot Camp aims to accelerate the steep learning curve 
that many new R users encounter, and will introduce 
skills for effectively using R in a research environment.
Effective use of R can greatly enhance
research productivity and facilitate successful communication of results
through high-quality data visualization. 

\section*{Requirements}
The most important requirement of this workshop is time and a willingness to
work.  Participants are expected to attend all sessions and spend at least 1 hour/day
on individual assignments.

Prior experience with R or a spreadsheet program like Excel is helpful but not
required.  Participants should be comfortable with basic
computer use, and must have their own computer.  Prior to the workshop, participants
must install R and Rstudio and obtain a dataset for workshop use.
Students are encouraged to work in small groups on a shared dataset.

\clearpage

\section*{Schedule}
D=Day, S=Session.  Sessions and labs are 1.5 hours.
\vspace{-5mm}
\subsection*{D1S1}
\begin{itemize}
    \item  Initial Evaluations
    \item  Syllabus, expectations
    \item  Review assignment, intro to packages and help
    \item  Introductions, datasets, research goals
    \item  Introduction to Rstudio - working with scripts, panes, shortcuts
    \item  Reading in data
\end{itemize}
\subsection*{Break: 30 minutes}
\subsection*{D1S2}
\begin{itemize}
    \item  Quiz, Questions, Review
    \item  What is a data.frame?
    \item  Indexing: Columns and Rows
    \item  Columns are vectors: mode 
    \item  Subsetting
    \item  Lab assignment: discuss datasets, identify 3 possible groups
\end{itemize}
\subsection*{Lunch}
\subsection*{Lab}
\subsection*{D1S3}
\begin{itemize}
    \item Choose groups
    \item Intro to factors
    \item Advanced data - Joins
    \item Intro to knitr
\end{itemize}

\clearpage
\subsection*{D2S1}
\begin{itemize}
    \item Quiz, Questions, Review
    \item Functions
    \item Vectorization
\end{itemize}
\subsection*{Break}
\subsection*{D2S2}
\begin{itemize}
    \item Quiz, Questions, Review
    \item Subsetting
    \item Marginal and conditional
    \item ddply: Functions in functions 
    \item Lab assignment: identify 3 questions that your data can be used to answer.
Sketch 1 graph for each question, including axis labels.
\end{itemize}
\subsection*{Lunch}
\subsection*{Lab}
\subsection*{D2S3}
\begin{itemize}
    \item Plotting with ggplot: aesthetics and layers
    \item Advanced data management: reshape2
\end{itemize}

\clearpage
\subsection*{D3S1}
\begin{itemize}
    \item Quiz, Questions, Review
    \item Scoping and namespaces
    \item Flow control
    \item Debugging and finding help
\end{itemize}

\subsection*{D3S2}
\begin{itemize}
    \item Quiz, Questions, Review
    \item Advanced plotting: facets, fine-tuning graphics
    \item Lab assignment: work on polishing figures 
\end{itemize}

\subsection*{Lunch}
\subsection*{Lab}
\subsection*{D3S3}
\begin{itemize}
    \item CRAN, packages, statistics
    \item Presentations: 10 min, 3 figures / group
    \item Final Review: where to go from here?
\end{itemize}

\clearpage
\section*{Assignments: Due at beginning of marked day}
\subsection*{Day 1}
\begin{itemize}
    \item Install plyr and reshape2
    \item Intro to help pages
    \item Identify the arguments of \texttt{rnorm}
    \item Identify the arguments of \texttt{sample}. Run the first example of
\texttt{sample}, and explain the results.
\end{itemize}

\subsection*{Day 2}
Use the provided knitr template to do the following:
\begin{itemize}
    \item Read in data
    \item Subset, summarize
    \item Choose at least 2 grouping variables in your data.set.  Use
\texttt{ddply} to compute marginal summaries over each of these variables
separately, and both together. 
    \item Describe your results in words.
\end{itemize}

\subsection*{Day 3}
\begin{itemize}
    \item Use knitr and ggplot to create 3 figures, including correct axis
labels, legends, and an informative caption.  One figure should
use facets, and one figure should include at least 3 aesthetics.  
    \item In your knitr document, explain your choice of aesthetics.
\end{itemize}

\end{document}
